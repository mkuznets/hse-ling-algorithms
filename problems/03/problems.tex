\documentclass[12pt,a4paper]{report}
% \textheight = 30cm
% \voffset = -36pt
\footskip = 0cm

\usepackage{cmap}
\usepackage{type1ec}
\usepackage[T2A]{fontenc}
\usepackage[utf8]{inputenc}
\usepackage[russian]{babel}

\usepackage{amsmath,amstext,amssymb}
\usepackage{fullpage}

\usepackage{enumitem}
\usepackage{ifpdf}
\ifpdf
  \usepackage[pdftex]{graphicx}
  \usepackage[pdftex,unicode,bookmarks=false]{hyperref}

  \pdfminorversion=5
  \pdfcompresslevel=9
  \pdfobjcompresslevel=9
\fi

\pagestyle{empty}

% Константа
\def\const{\mathop{\mathrm{const}}\nolimits}

\renewcommand{\thesection}{\arabic{section}}
\renewcommand{\thesubsection}{}

\begin{document}

\begin{center}
\textbf{\large{Теория алгоритмов. Листок 3}}\\
BSc, компьютерная лингвистика, НИУ ВШЭ\\
Выдан: 19 марта 2018\\
\end{center}

{\bf N.B.: каждая задача подразумевает наличие подробного решения и/или доказательства, предшествующего ответу. В противном случае задача не будет засчитана.}

\begin{enumerate}
  \item\,[{\tt recurrences}] Для каждой рекуррентной формулы получите асимптотическое решение с помощью Master Theorem, либо укажите, что теорему нельзя применить (и почему).

  \begin{enumerate}
    \item $T(n) = 2T(n/4) + \lg n$
    \item $T(n) = 3T(n/4) + n \lg n$
    \item $T(n) = T(n/2) + T(n/4) + n^2$
    \item $T(n) = 7T(n/2) + n^2$
    \item $T(n) = 3T(n/3) + n/2$
  \end{enumerate}

  \item\,[{\tt long\_mult}] Оцените временную сложность умножения {\em столбиком} двух $n$-разрядных целых чисел. Сложение и умножение одноразрядных чисел следует считать выполняющимися за константное время.

  {\em (В этой и следующей задачах предполагается, что мы имеем дело только с десятичной системой счисления.)}

  \item\,[{\tt karatsuba}] \href{https://en.wikipedia.org/wiki/Karatsuba_algorithm}{Алгоритм Карацубы} позволяет перемножить целые числа быстрее, чем с помощью умножения столбиком. Пусть даны два $n$-разрядных числа $x$ и $y$. Выберем $m<n$, например $m = \lceil n/2 \rceil$. Разложим множители:
  $$
  \begin{gathered}
  x = x_1 \cdot 10^m + x_0\\
  y = y_1 \cdot 10^m + y_0
  \end{gathered}
  $$

  Запишем произведение:
  $$
  \boxed{xy} = (x_1 \cdot 10^m + x_0)(y_1 \cdot 10^m + y_0)
       \equiv \boxed{z_2 \cdot 10^{2m} + z_1 \cdot 10^{m} + z_0}
  $$%
  %
  где%
  $$
  \begin{gathered}
  z_0 = x_0 y_0   ~~~~~~~~~~  z_2 = x_1 y_1\\
  z_1 = x_1 y_0 + x_0 y_1 \equiv (x_1 + x_0)(y_1 + y_0) - z_2 - z_0
  \end{gathered}
  $$%

  Таким образом, задача умножения $n$-разрядных чисел сведена к трём умножениям $n/2$-разрядных чисел и операциям сложения. Рекурсивно применяем к этим умножениям алгоритм Карацубы, и получаем ответ.

  Получите асимптотику временной сложности этого алгоритма.
\end{enumerate}

\end{document}