\documentclass[12pt,a4paper]{report}
% \textheight = 30cm
% \voffset = -36pt
\footskip = 0cm

\usepackage{cmap}
\usepackage{type1ec}
\usepackage[T2A]{fontenc}
\usepackage[utf8]{inputenc}
\usepackage[russian]{babel}

\usepackage{amsmath,amstext,amssymb}
\usepackage{fullpage}

\usepackage{ifpdf}
\ifpdf
  \usepackage[pdftex]{graphicx}
  \usepackage[pdftex,unicode,bookmarks=false]{hyperref}

  \pdfminorversion=5
  \pdfcompresslevel=9
  \pdfobjcompresslevel=9
\fi

\pagestyle{empty}


\renewcommand{\thesection}{\arabic{section}}
\renewcommand{\thesubsection}{}

\begin{document}

\begin{center}
\textbf{\LARGE{Теория алгоритмов. Листок 1}}\\
BSc, компьютерная лингвистика, НИУ ВШЭ\\
Выдан: 20 января 2016\\
Дедлайн: 27 января 00:00\\
\end{center}

% == Лекция 2 ==
% Абстрактные типы данных. 
% Список и его реализации: массив, связный список (одно- и двусвязный), динамический массив. Бинарный поиск в массиве. list в Python.
% Стек, очередь, двухсторонняя очередь. Их реализация через динамический массив и двусвязный список. dequeue в Python.
% Коллекции с быстрыми добавлением, удалением и поиском.
% Словарь. Реализация через бинарное дерево и хэш-таблицы. Разрешение коллизий: цепочки, открытая адресация (линейное пробирование).
% Множества. Реализация через хэш-таблицы

% == Семинар 2 ==
% in-place алгоритмы
% Цикл в связном списке
% Многомерные массивы (vs реализации через вложенные списки)
% Метод бисекции в математике, вычисление корня
% Удалить элемент из связного списка
% Стек для рекурсии и проверки скобочного выражения
% LRU cache (обсуждение, реализация - в задачи)
% 


% 2 (1)
% 3


% == Листок 2 ==

% Массивы, in-place и бинарный поиск
% (2) 283. Move Zeroes
% (4) 153. Find Minimum in Rotated Sorted Array

% Связные списки
% (2) 237. Delete Node in a Linked List
% (2) 141. Linked List Cycle
% (3) 206. Reverse Linked List


% Стеки и очереди
% (2) 20. Valid Parentheses
% (2) 225. Implement Stack using Queues

% Словари
% (3) 146. LRU Cache
% (5) 146. LRU Cache
% 




\section{Массивы, in-place алгоритмы и бинарный поиск}

\begin{enumerate}

  \item\,[2] \href{https://leetcode.com/problems/move-zeroes/}{Move Zeroes}\\
  Напишите функцию, принимающую массив целых чисел, которая перемещает нули в конце массива, сохраняя порядок ненулевых элементов.
  Например, если исходный массив равен \texttt{[0, 1, 0, 3, 12]}, после применения функции он должен быть равен \texttt{[1, 3, 12, 0, 0]}.

  \begin{verbatim}
class Solution(object):
    def moveZeroes(self, nums):
        """
        :type nums: List[int]
        :rtype: void Do not return anything, modify nums in-place instead.
        """
  \end{verbatim}     


\end{enumerate}


\section{Связные списки}


\section{Стеки и очереди}


\section{Словари}

\end{document}