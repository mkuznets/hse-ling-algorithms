\documentclass[12pt]{extarticle}
%Russian-specific packages
%--------------------------------------
\usepackage[T2A]{fontenc}
\usepackage[utf8]{inputenc}
\usepackage[russian]{babel}
%--------

\title{hw3_algorithms}
\author{83210 }
\date{February 2017}


\usepackage{geometry,nopageno}
\geometry{a5paper,left=1cm,right=1cm,top=1cm,bottom=1cm}
\usepackage{cmap, type1ec}
\usepackage{verbatim,nameref}
\usepackage{amsmath,amsthm,amstext,amssymb,amscd,
            mathtools,mathrsfs,dsfont}
            
\newtheorem*{problem}{Задача}

\begin{document}

\begin{problem}[1]
{\em Инверсией} в перестановке $\pi$ называется пара индексов $(i,j)$ такая что $i<j$, но $\pi_i>\pi_j$. Докажите, что в перестановке из $n$ элементов может быть не более $n(n-1)/2$ инверсий. В какой перестановке количество инверсий ровно $n(n-1)/2$?
\end{problem}

\begin{proof}

 У нас есть перестановка из n элементов. Чтобы понять, сколько в ней инверсий: \begin{enumerate}
    \item Hайдем наименьшее число в наборе.
    \item Посмотрим, сколько элементов стоит левее. Максимально их может быть (длина набора - 1), если набор отсортирован от большего к меньшему. Минимально - 0, если набор отсортирован от меньшего к большему.
    \item Удалим наименьший элемент из набора. 
    \item Повторяем предыдущие шаги, пока в нем не останется 1 элемент.
\end{enumerate}
    
Таким образом на i-ой итерации длина набора равна $n-i$, всего итераций будет $n-1$. То есть максимальное число инверсий в наборе будет $\sum_{i=1}^{n-1} (n-i) = \frac{1}{2}n(n-1)$ в случе, когда набор отсортирован от большего к меньшему.

\end{proof}

\end{document}
