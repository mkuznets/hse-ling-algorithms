
\begin{proof}

 У нас есть перестановка из n элементов. Чтобы понять, сколько в ней инверсий: \begin{enumerate}
    \item Hайдем наименьшее число в наборе.
    \item Посмотрим, сколько элементов стоит левее. Максимально их может быть (длина набора - 1), если набор отсортирован от большего к меньшему. Минимально - 0, если набор отсортирован от меньшего к большему.
    \item Удалим наименьший элемент из набора. 
    \item Повторяем предыдущие шаги, пока в нем не останется 1 элемент.
\end{enumerate}
    
Таким образом на i-ой итерации длина набора равна $n-i$, всего итераций будет $n-1$. То есть максимальное число инверсий в наборе будет $\sum_{i=1}^{n-1} (n-i) = \frac{1}{2}n(n-1)$ в случе, когда набор отсортирован от большего к меньшему.

\end{proof}
