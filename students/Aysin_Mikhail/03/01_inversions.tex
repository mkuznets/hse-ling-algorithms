\documentclass[12pt]{extarticle}
\usepackage{geometry,nopageno}
\geometry{a5paper,left=1cm,right=1cm,top=1cm,bottom=1cm}
\usepackage{cmap, type1ec}
\usepackage[T2A]{fontenc}
\usepackage[utf8]{inputenc}
\usepackage[russian]{babel}

\usepackage{verbatim,nameref}
\usepackage{amsmath,amsthm,amstext,amssymb,amscd,
            mathtools,mathrsfs,dsfont}
            
\newtheorem*{problem}{Задача}

\begin{document}

\begin{problem}[1]
{\em Инверсией} в перестановке $\pi$ называется пара индексов $(i,j)$ такая что $i<j$, но $\pi_i>\pi_j$. Докажите, что в перестановке из $n$ элементов может быть не более $n(n-1)/2$ инверсий. В какой перестановке количество инверсий ровно $n(n-1)/2$?
\end{problem}

\begin{proof}

Предположим, что существует перестановка, в которой для каждой пары индексов $i, j$ выполнена инверсия, т.е. $i<j$, $\pi_i>\pi_j$.
Сколько всего пар индексов в перестановке из $n$ элементов? Пара индексов - это {\em неупорядоченный} набор из двух элементов, выбираемый без повторений среди множества из $n$ номеров, поэтому это считается как число сочетаний $$C_n^2 = \frac{n(n-1)}{2}$$Больше пар быть не может, значит в перестановке с максимальным числом инверсий будет ровно $n(n-1)/2$ инверсий. Существуют также перестановки с меньшим числом инверсий. 

Перестановка с максимальным числом инверсий единственна, вот она:
$$
\begin{pmatrix}
    1 & 2 & 3 & ... & n-2 & n-1 & n \\
    n & n-1 & n-2 & ... & 3 & 2 & 1
\end{pmatrix}
$$



\end{proof}

\end{document}
