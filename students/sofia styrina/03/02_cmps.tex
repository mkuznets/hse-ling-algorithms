%% LyX 2.1.1 created this file.  For more info, see http://www.lyx.org/.
%% Do not edit unless you really know what you are doing.
\documentclass[english]{article}
\usepackage[T1]{fontenc}
\usepackage[latin9]{inputenc}
\usepackage{esint}
\usepackage{babel}
\begin{document}

\section*{Problem 2}

$log_{2}(n!)\ge c\times n\times log_{2}(n)$

\medskip{}


Let's multiply each side of the inequality by $\ln2>0$

$\ln(n!)\ge c\times n\times\ln(n)$

Let's divide both sides by n to obtain

$\frac{\ln(n!)}{n}\ge c\times\ln(n)$

\medskip{}


$\frac{1}{n}\ln(n!)=\frac{1}{n}\sum_{k=1}^{n}\ln(k)=\frac{1}{n}\sum_{k=1}^{n}\ln(\frac{k}{n})+\frac{1}{n}\sum_{k=1}^{n}\ln(n)=\frac{1}{n}\sum_{k=1}^{n}(\frac{k}{n})+\frac{1}{n}n\ln(n)=\frac{1}{n}\sum_{k=1}^{n}\ln(\frac{k}{n})+\ln(n)$

\medskip{}


$\frac{1}{n}\sum_{k=1}^{n}\ln(\frac{k}{n})=\left(x_{k}=\frac{k}{n}\in(0,1];\Delta x=\frac{1}{n}\right)=$$\sum_{k=1}^{n}\ln(x_{k})\times\Delta x\approx\int_{0}^{1}\ln xdx$

where the last equality uses the definition of the Riemann integral.

\medskip{}


$\int_{0}^{1}\ln xdx=1\times\ln(1)-\lim_{x\rightarrow0}x\times\ln(x)-\int_{0}^{1}x\times\frac{1}{x}dx$

\medskip{}


Applying the L'Hospital's Rule,

$\lim_{x\rightarrow0}x\times\ln(x)=\lim_{x\rightarrow0}\frac{\ln x}{1/x}=\lim_{x\rightarrow0}\frac{1/x}{(-1/x^{2})}=0$

\medskip{}


Hence, $\frac{1}{n}\ln(n!)=\ln n-0-\int_{0}^{1}dx=\ln n-1$

\medskip{}


Plugging the last expression into the inequality, one obtains

$\ln n-1\ge c\ln n$

\medskip{}


Dividing both sides of the inequality by n yields:

$\ln n-1\ge c\ln n$, which is equivalent to $\ln n\ge\frac{1}{1-c}$

\medskip{}


Let's pick the value of c in $(0,1)$. For example, $c=0,5$

$\ln n\ge\frac{1}{1-0,5}=2$, or $n\ge e^{2}$.

\medskip{}


For any $n\ge9$, the last inequality holds and so does the original
inequality $log_{2}(n!)\ge c\times n\times log_{2}(n)$

\medskip{}


Q.E.D.
\end{document}
